%------------------------------------
% Dario Taraborelli
% Typesetting your academic CV in LaTeX
%
% URL: http://nitens.org/taraborelli/cvtex
% DISCLAIMER: This template is provided for free and without any guarantee 
% that it will correctly compile on your system if you have a non-standard  
% configuration.
% Some rights reserved: http://creativecommons.org/licenses/by-sa/3.0/
%------------------------------------

%!TEX TS-program = xelatex
%!TEX encoding = UTF-8 Unicode

\documentclass[11pt, a4paper]{article}
\usepackage{fontspec} 
\usepackage{csquotes}
\usepackage{polyglossia}
\setmainlanguage{french}

% DOCUMENT LAYOUT
\usepackage{geometry} 
\geometry{a4paper, textwidth=5.5in, textheight=8.5in, marginparsep=7pt, marginparwidth=.6in}
\setlength\parindent{0in}

% FONTS
\usepackage[usenames,dvipsnames]{color}
\usepackage{xunicode}
\usepackage{xltxtra}
\defaultfontfeatures{Mapping=tex-text}
\setromanfont [Ligatures={Common}, Numbers={OldStyle}, Variant=01]{Linux Libertine O}
\setmonofont[Scale=0.8]{Monaco}

\RequirePackage{xcolor}
\definecolor{darkgray}{HTML}{333333}
\definecolor{gray}{HTML}{4D4D4D}
\definecolor{lightgray}{HTML}{999999}

% ---- CUSTOM COMMANDS
\chardef\&="E050
\newcommand{\html}[1]{\href{#1}{\scriptsize\textsc{[html]}}}
\newcommand{\pdf}[1]{\href{#1}{\scriptsize\textsc{[pdf]}}}
\newcommand{\doi}[1]{\href{#1}{\scriptsize\textsc{[doi]}}}
% ---- MARGIN YEARS
\usepackage{marginnote}
\newcommand{\amper{}}{\chardef\amper="E0BD }
\newcommand{\years}[1]{\marginnote{\scriptsize #1}}
\renewcommand*{\raggedleftmarginnote}{}
\setlength{\marginparsep}{7pt}
\reversemarginpar

% HEADINGS
\usepackage{sectsty} 
\usepackage[normalem]{ulem} 
\sectionfont{\mdseries\upshape\Large}
\subsectionfont{\mdseries\scshape\normalsize} 
\subsubsectionfont{\mdseries\upshape\large} 

% PDF SETUP
% ---- FILL IN HERE THE DOC TITLE AND AUTHOR
\usepackage[xetex, bookmarks, colorlinks, breaklinks, 
% ---- FILL IN HERE THE TITLE AND AUTHOR
	pdftitle={Joël Gombin -- CV},
	pdfauthor={Joël Gombin},
	pdfproducer={http://nitens.org/taraborelli/cvtex}
]{hyperref}  
\hypersetup{linkcolor=blue,citecolor=blue,filecolor=black,urlcolor=MidnightBlue} 

% BIBLIOGRAPHY
\RequirePackage[style=verbose, maxnames=99, sorting=ynt, backend=biber]{biblatex}
\addbibresource{publis.bib}
\DeclareFieldFormat[article]{title}{#1\par}
\DeclareFieldFormat[book]{title}{#1\par}
\DeclareFieldFormat[incollection]{title}{#1\par}
\DeclareFieldFormat[inproceedings]{title}{#1\par}
\DeclareFieldFormat[report]{title}{#1\par}
\DeclareFieldFormat[misc]{title}{#1\par}
\DeclareFieldFormat[thesis]{title}{#1\par}
\DeclareFieldFormat{year}{\years{#1}}


\DeclareBibliographyDriver{article}{%
\begin{minipage}{\textwidth}
 \printfield{year}%
 \href{\thefield{url}}{\printfield{title}}%
 \newblock%
  \printnames{author}%
  \par%
  \newblock%
  {%
    \footnotesize\addfontfeature{Color=darkgray}\itshape%
    \usebibmacro{journal}%
    \setunit{\space}%
    \printfield{volume}%
    \setunit{\space}%    
    \emph{\printfield{number}}%
    \setunit{\space}%
    \printfield{pages}%
    }
  \par\vspace{0.3\baselineskip}
\end{minipage}
}

\DeclareBibliographyDriver{book}{%
\begin{minipage}{\textwidth}
 \printfield{year}%
 \href{\thefield{url}}{\printfield{title}}%  
 \newblock%
  \printnames{editor}\addfontfeature{Color=darkgray}\ (dir.)%
  \par%
  \newblock%
  {%
    \footnotesize\addfontfeature{Color=darkgray}\itshape%
    \printfield{series}%
    \setunit{,\space}%
    \printlist{publisher}%
    \setunit{\space}%
    \printlist{location}%
    }
  \par\vspace{0.3\baselineskip}
\end{minipage}
}

\DeclareBibliographyDriver{incollection}{%
\begin{minipage}{\textwidth}
  \printfield{year}%
 \href{\thefield{url}}{\printfield{title}}%
  \newblock%
  \printnames{author}%
  \par%
  \newblock%
  {%
    \footnotesize\addfontfeature{Color=darkgray}\itshape%
    \printnames{editor}%
    \setunit{ (dir.)\space}%
    \usebibmacro{booktitle}%
    \setunit{\space}%
    \printlist{location}%
    \setunit{\space}%
    \printfield{pages}%    
    }
  \par\vspace{0.3\baselineskip}
\end{minipage}
}

\DeclareBibliographyDriver{inproceedings}{%
\begin{minipage}{\textwidth}
  \printfield{year}%
 \href{\thefield{url}}{\printfield{title}}%
  \newblock%
  \printnames{author}%
  \par%
  \newblock%
  {%
    \footnotesize\addfontfeature{Color=darkgray}%
    \printfield{booktitle}%
    \setunit{\addcomma\space}%
    \printlist{location}%
    \newunit%
  }
  \par\vspace{0.3\baselineskip}
\end{minipage}
}

\DeclareBibliographyDriver{misc}{%
\begin{minipage}{\textwidth}
  \printfield{year}%
 \href{\thefield{url}}{\printfield{title}}%  
 \newblock%
  \printnames{author}%
  \par%
  \newblock%
  {%
    \footnotesize\addfontfeature{Color=darkgray}\itshape%
    \printfield{booktitle}%
    \setunit*{\addcomma\space}%
    \printfield{note}%
    \setunit*{\addcomma\space}%
    \printfield{year}%
    \setunit{\addcomma\space}%
    \printlist{location}%
    \newunit%
  }
  \par\vspace{0.3\baselineskip}
\end{minipage}
}

\DeclareBibliographyDriver{report}{%
\begin{minipage}{\textwidth}
   \printfield{year}%
 \href{\thefield{url}}{\printfield{title}}%
   \newblock%
  \printnames{author}%
  \par%
  \newblock%
  {%
    \footnotesize\addfontfeature{Color=darkgray}\itshape%
    \printnames{editor}
    \setunit{\space}%
    \usebibmacro{booktitle}%
    \setunit{\space}%
    \printlist{location}%
    \setunit{\space}%
    \printfield{institution}%
    \newunit%
  }
  \par\vspace{0.3\baselineskip}
\end{minipage}
}

\DeclareBibliographyDriver{thesis}{%
\begin{minipage}{\textwidth}
   \printfield{year}%
 \href{\thefield{url}}{\printfield{title}}%
   \newblock%
  \printnames{author}%
  \par%
  \newblock%
  {%
    \footnotesize\addfontfeature{Color=darkgray}\itshape%
    \printfield{type}
    \setunit{\space}%
    \printlist{location}%
    \setunit{\space}%
    \printfield{school}%
    \newunit%
  }
  \par\vspace{0.3\baselineskip}
\end{minipage}
}


\DeclareNameFormat{author}{%
  \small\addfontfeature{Color=darkgray}%
  \ifblank{#3}{}{#3\space}#1%
  \ifthenelse{\value{listcount}<\value{liststop}}
    {\addcomma\space}
    {}%
}

\DeclareNameFormat{editor}{%
  \small\addfontfeature{Color=darkgray}%
  \ifblank{#3}{}{#3\space}#1%
  \ifthenelse{\value{listcount}<\value{liststop}}
    {\addcomma\space}
    {}%
}


\newcommand{\printbibsection}[3]{
  \begin{refsection}
    \nocite{*}
    \printbibliography[sorting=chronological, type={#1}, keyword={#2}, title={#3}, heading=subbibliography]
  \end{refsection}
}

\newcommand{\printbibsectionNK}[2]{
  \begin{refsection}
    \nocite{*}
    \printbibliography[sorting=chronological, type={#1}, title={#2}, heading=subbibliography]
  \end{refsection}
}

\DeclareSortingScheme{chronological}{
  \sort[direction=descending]{\field{year}}
  \sort[direction=descending]{\field{month}}
}

\urlstyle{same}



% DOCUMENT
\begin{document}
{\LARGE Joël Gombin}\\[1cm]
 CURAPP\\
Université de Picardie Jules Verne\\
Faculté de droit et de science politique\\
Pôle universitaire Cathédrale\\
10, placette Lafleur\\
BP 2716\\
80027 Amiens cedex 1\\
France\\[.2cm]
Téléphone: +33~6~61~55~22~41\\
E-mail: \href{mailto:joel.gombin@u-picardie.fr}{joel.gombin@u-picardie.fr}\\
\textsc{url}: \href{http://www.joelgombin.fr}{\url{http://www.joelgombin.fr}}\\[.2cm]
Né le 15 janvier 1983 à Paris 17\textsuperscript{e}, France\\
Nationalité : française

\vfill

%%\hrule
\section*{Thèmes de recherche}
Sociologie électorale -- Front national -- Mondes agricoles -- Conspirationnisme

%%\hrule
\section*{Situation actuelle}
\years{2011-2013} Doctorant en science politique, \textsc{curapp}, Université de Picardie-Jules Verne.

Direction : Patrick Lehingue (PR, \textsc{Upjv}) et Christophe Traïni (PR, \textsc{Iep} Aix).

%%\hrule
%\section*{Areas of specialization}
 %Physics • Relativity theory

%%\hrule
\section*{Postes occupés}
\noindent
\years{2011-2011}Chef de projet, association Voiture \& co, Paris\\
\years{2008-2010}Attaché temporaire d'enseignement et de recherche (\textsc{Ater}) en science politique, Faculté de droit et de science politique, Université de Picardie Jules Verne.\\
\years{2005-2008}Allocataire de recherche, \textsc{Cspc} -- Institut d'études politiques, Aix-en-Provence\\

%\hrule
\section*{Formation}
\noindent
\years{2005-2013}Thèse en cours sur le sujet : « Configurations locales et construction sociale des électorats. Etude comparative des votes FN en région PACA », \textsc{Iep} d'Aix-en-Provence puis Université de Picardie-Jules Verne.
\years{2005}Master de politique comparée, mention \enquote{Europe méditerranéenne}, \textsc{Iep} d'Aix-en-Provence. \\
\years{2004}Diplôme de l'\textsc{Iep} d'Aix-en-Provence, section \enquote{science politique}. Mention Bien\\
Licence de droit, Université Paul-Cézanne. Mention Assez bien\\
\years{2000}Baccalauréat, série \textsc{es}, mention \textsc{tb} avec les Félicitations du jury\\


%\hrule
\section*{Publications}

 \printbibsection{article}{comité de lecture}{Articles publiés dans des revues à comité de lecture}
 \printbibsection{article}{sans comité de lecture}{Articles publiés dans des revues sans comité de lecture}
 \printbibsectionNK{book}{Direction d'ouvrage}
 \printbibsectionNK{incollection}{Chapitres d'ouvrage} 
 \printbibsectionNK{inproceedings}{Communications}
 \printbibsectionNK{report}{Rapport de recherche}
 \printbibsection{article}{recension}{Recensions d'ouvrage}
 \printbibsectionNK{thesis}{Travaux universitaires}

\section*{Enseignements}


\begin{minipage}{\linewidth}
\years{2010-2011}\textbf{Université Paris-\textsc{i} Panthéon Sorbonne}\\
Chargé d'enseignement\\
Licence 1 de droit\\
\emph{Travaux dirigés \enquote{Introduction à la science politique}}\\
\end{minipage}

\begin{minipage}{\linewidth}
\years{2008-2010}\textbf{Université de Picardie Jules Verne}\\
ATER\\
Licence 1 de droit\\
\emph{Travaux dirigés \enquote{Droit constitutionnel}}\\
\emph{Travaux dirigés \enquote{Introduction à la science politique}}\\
Licence 3 de science politique\\
\emph{Travaux dirigés \enquote{Théories sociologiques}}\\
Master 1 de science politique\\
\emph{Travaux dirigés \enquote{Méthodes des sciences sociales}}\\
\emph{Travaux dirigés \enquote{Sociologie électorale}}\\
\end{minipage}

\begin{minipage}{\linewidth}
\years{2007-2009}\textbf{Université Paris-\textsc{xiii}}\\
Chargé d'enseignement\\
Licence 1 de sociologie politique\\
\emph{Travaux dirigés \enquote{Droit et institutions publiques}}\\
Licence 3 de sociologie politique\\
\emph{Travaux dirigés \enquote{Sociologie de l'État}}
\end{minipage}

\begin{minipage}{\linewidth}
\years{2007-2008}\textbf{Institut régional du Travail social, Neuilly-sur-Marne}\\
Chargé d'enseignement\\
Licence 2 d'administration économique et sociale -- formation d'assistant social\\
\emph{Travaux dirigés \enquote{Politique sociale}}
\end{minipage}

\begin{minipage}{\linewidth}
\years{2006-2007}\textbf{Université Lyon-\textsc{ii}}\\
Chargé d'enseignement\\
Licence 3 de science politique\\
\emph{Travaux dirigés \enquote{Processus de décision publique}}
\end{minipage}

\begin{minipage}{\linewidth}
\years{2004-}\textbf{Prépa \textsc{Iep} Ambition Réussite}\\
Enseignant\\
\emph{Histoire contemporaine, méthodologie, actualité, ouvrage, culture générale}
\end{minipage}



%\hrule
\section*{Animation de la recherche et recherches collectives}
\years{2011-}Participation au projet de recherche \enquote{Sociologie politique de l'élection} (SPEL), dirigé par Daniel Gaxie et Patrick Lehingue
\years{2011}Organisation, avec Jean Rivière, de la section thématique 34 \enquote{Géographie et sociologie électorales : duel ou duo ? Actualité et avenir d'une concurrence/collaboration scientifique} au Congrès de l'Association française de science politique, Strasbourg\\
\years{2009}Organisation de la journée d'étude \enquote{L'importance du niveau \enquote{méso} dans l'étude du succès de l'extrême droite},  \textsc{Upjv-Curapp}-Université d'Anvers, Amiens \\
\years{2006-2008}Participation au projet de recherche subventionné par l'\textsc{anr} \enquote{Pour une analyse écologique des comportements électoraux (\textsc{paece})}, dirigé par le Pr Jean-Yves Dormagen (coordination d'un terrain marseillais) \\
\years{2006-2007}Coordination au sein du CSPC du groupe de recherche \enquote{Elections 2007} et animation d’unséminaire méthodologique \enquote{Etudes électorales : méthodes, théories, données} \\

%hrule
\section*{Valorisation de la recherche}
Interventions régulières dans de nombreux médias régionaux et nationaux, sur le vote FN et le vote des agriculteurs\\
Diverses conférences \\

%hrule
\section*{Engagements professionnels et associatifs}
\begin{minipage}{\linewidth}\years{2004}Vice-président étudiant du Conseil d'administration de l'\textsc{Iep} d'Aix-en-Provence\end{minipage}\\
\years{2012-}Création, avec Maïeul Rouquette, d'une association \enquote{\LaTeX\ en sciences humaines et sociales}
\years{2008-2010}Président de l'association \emph{Les nouvelles Éditions universitaires}, éditrice du \emph{Mensuel de l'Université} et de \url{www.universitemag.fr}\\
\years{2006-2008}Rédacteur et membre du comité de direction du \emph{Mensuel de l'Université}\\
\years{2006-2008}Vice-président, puis président, de l'Association nationale des candidats aux métiers de la science politique (\textsc{ancmsp})\\
\years{2005-2008}Membre actif, puis secrétaire général, de la Confédération des jeunes chercheurs\\


%\vspace{1cm}
\vfill{}
%\hrulefill

\begin{center}
{\scriptsize  Dernière mise à jour : \today\- •\- 
% ---- PLEASE LEAVE THIS BACKLINK FOR ATTRIBUTION AS PER CC-LICENSE
Mise en page en \href{http://nitens.org/taraborelli/cvtex}{
\fontspec{Times New Roman}\XeTeX }\\
% ---- FILL IN THE FULL URL TO YOUR CV HERE
\href{http://www.joelgombin.fr/cv}{http://www.joelgombin.fr/cv}}
\end{center}

\end{document}